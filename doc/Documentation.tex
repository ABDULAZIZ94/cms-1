\documentclass[a4paper,8pt]{amsart}

\usepackage[osf,sc]{mathpazo}
\usepackage[a4paper]{geometry}
\usepackage{fullpage}
\usepackage{multicol}

\newcommand{\CMS}{\textsc{cms}}

\newcommand{\FS}{\textsc{fs}}
\newcommand{\CDBS}{\textsc{cdbs}}
\newcommand{\CWS}{\textsc{cws}}
\newcommand{\ES}{\textsc{es}}
\newcommand{\WS}{\textsc{ws}}
\newcommand{\LWS}{\textsc{lws}}
\newcommand{\RWS}{\textsc{rws}}

\newenvironment{squishlist}{%
  \begin{list}{\textbullet}%
    { \setlength{\itemsep}{0pt}%
      \setlength{\parsep}{3pt}%
      \setlength{\topsep}{3pt}%
      \setlength{\partopsep}{0pt}%
      \setlength{\leftmargin}{1.5em}%
      \setlength{\labelwidth}{1em}%
      \setlength{\labelsep}{0.5em} }%
}{\end{list}}

\newcommand{\id}[1]{\texttt{#1}}

\title{A Contest Management System for programming contests}
\author{Matteo Boscariol, Massimo Cairo, Stefano Maggiolo, Giovanni
  Mascellani}
\date{\today}


\begin{document}

\maketitle
\tableofcontents

\begin{multicols}{2}

  \section{Introduction}

  When organizing a programming contest, there are three main stages:
  \begin{squishlist}
  \item the first is to develop all the data that the assigned tasks
    need (i.e., the statement, solutions, testcases, information on
    how to grade submissions, etc.);
  \item the second, that happens when the contest is onsite, is to
    properly configure the computers that the contestants are going to
    use during the contest, in particular with respect to network
    security;
  \item the third is to manage the actual contest (accepting and
    grading submissions, give feedback on them, display a live
    ranking, etc.).
  \end{squishlist}

  The aim of this project is to give a good answer to the third
  problem. We want to develop a contest management system that is
  secure, extendable, adaptable to different situations, and easy to
  use.

  \section{General structure of the system}

  The system is organized in a modular way, with different services
  running (potentially) on different computers, and providing
  extendability via service replications on several computers.

  The system is composed of the following functional pieces:
  \begin{squishlist}
  \item File Storage (\FS{}), that stores files that can be accessed from
    the rest of the system;
  \item CouchDB Server (\CDBS{}), that stores the configuration and the
    state of the contest;
  \item Contest Web Server (\CWS{}), that interacts with the contestants;
  \item Evaluation Server (\ES{}), that grades the submissions and keeps
    track of the state of the contest;
  \item Worker Server (\WS{}), that actually runs the submissions;
  \item Ranking Web Server (\RWS{}), that expose in real time to the
    world what we want the world to know about the contest;
  \item Log Web Server (\LWS{}), that expose to the contest managers
    everything about the contest, and gives the possibility to change
    on the run some configurations.
  \end{squishlist}

  One is not forced to run everything; here are some examples.
  \begin{squishlist}
  \item Bare: there are only a \FS{}, a \CDBS{}, and a \CWS{}; the
    system would accept submissions, but is unable to give feedback to
    the users and to the world, lacking something that evaluate the
    submissions.
  \item Minimal: add an \ES{} and a \WS{}; now we can evaluate
    submissions, so the users are informed of the state of the
    submissions.
  \item Complete: add a \RWS{} and a \LWS{}, in order to show the live
    ranking and controls locally if anything strange is happening.
  \item Redundant: add several other \WS{}'s to manage big contest
    without having long evaluation queues.
  \end{squishlist}

  \subsection{Interaction with the task development}

  The development of the tasks is completely detached from the
  system. One can use some directory structure, ease the development
  with the usage of makefiles, common libraries, or other
  techniques. We believe that every contest manager has his own
  preferred way on how to proceed in this stage, but we want him to
  use our \CMS{} anyway.

  Therefore, there is only one point of contact between our \CMS{} and
  the actual structure on the contest manager's file system: anytime
  after the finalization of the tasks and before the start of the
  contest, the contest manager has to run an \emph{importer\/} that
  configures \CMS{} and fills \FS{} with the needed files.

  We wrote an example of importer that works for the structure used by
  the Italian IOI team. Converting this program to use other
  structures should be easy.

  A slightly more complicated issue is that different contest managers
  can use different type of tasks.

  \subsubsection{An example}

  Before IOI 2010 in Canada, the tasks assigned at the IOI were mainly
  ``batch'' tasks, where the contestants have to submit a stand-alone
  source file; there had been sporadic cases of ``output-only'' tasks,
  where the contestants have the input cases and need to submit only
  the outputs to these inputs, and of ``interactive'' tasks, where the
  contestants submit a source file that will be linked to a given
  library.

  From IOI 2010, the distinction between batch and interactive tasks
  vanished, being them united in the "programming" tasks. In all these
  tasks, the contestants have to submit one or more source files that
  implement one or more functions; these source files will be linked
  to a given source file that implements the I/O and calls the
  contestants' functions.

  Note that the programming tasks may need to be treated differently
  in the evaluation. For example, when the statement of the task says
  that the contestants have to implement two functions that cannot
  communicate between each other, the evaluation needs to ensure this.

  These difference can be implemented \emph{inside} the system, as
  plugins. We believe that we internal structure of a task is flexible
  enough to manage most task types.

  \section{Description of the services}

  We give here a detailed description of the capabilities and the
  current role of the services that compose \CMS{}.

  \subsection{Generalities}

  We distinguish two kind of services in \CMS{}: external services and
  local services. An external service is something that serves web
  pages; there are three of them: \CWS{}, that serves the contestants,
  \LWS{}, that serves the contest managers, and \RWS{}, that serves
  everyone else. A local service instead is not accessible from
  outside the system. Except \CDBS{}, which actually did not need any
  line of code from us, all else are local services.

  The services are written in Python; external services use Tornado to
  serve dynamic web pages, and nginx to serve static files and do the
  caching. All services communicate between each other using the
  \id{SimpleXMLRPCServer} library.

  \subsection{File Storage}

  File Storage is a simple service, but is essential to any
  contest. It is capable of accepting files, storing them and offering
  them when asked to.

  In particular, these are its uses:
  \begin{squishlist}
  \item before the contest, the importer ask \FS{} to store the tasks'
    data: the statement in PDF (needed by \CWS{}), inputs, outputs,
    source files that are going to be used during the evaluation of
    submissions (needed by \WS{});
  \item during the contest, after receiving a submission, \CWS{} ask
    \FS{} to store its associated files (being them outputs or source
    files); then, \WS{} needs these files to evaluate the submission.
  \end{squishlist}

  \subsection{CouchDB Server}

  CouchDB Server is not actually a service of \CMS{}, but no more that
  an istance of CouchDB. Therefore, it can stores \emph{documents\/},
  that for us are serializations of Python objects, where a reference
  to another object is translated to the CouchDB id of the referenced
  object.

  Its main use in \CMS{} is to store the contest's data, that is the
  specification of the contest, tasks, contestants, submissions. The
  documents in \CDBS{} may have links to files contained in \FS{}.

  \subsection{Contest Web Server}

  Contest Web Server is a Tornado application that serve (localized)
  web pages to the contestants. Here are the main functionality
  exposed to them.
  \begin{squishlist}
  \item View the tasks' statement and the attachments (that usually
    are example of implementations of the functions, or big example of
    test data).
  \item Submit a (partial) solution for a task. Since a task may
    require more than one file, if \CWS{} receives an archive (zip,
    tar, tar.gzip, or tar.bzip2), it unpack the archive and assign the
    files to the submission. It is a duty of the plugin specifying the
    task type to decide what happens when there are some files
    missing. For a programming task, it makes sense to reject the
    submission; for an output-only task, it makes sense to use the
    previously sent outputs. When a submission is received, \CWS{}
    informs \ES{} and ask it to evaluate the submission.
  \item View the state of submissions, in particular if they have been
    compiled and executed against the inputs that ensure that the
    submission is syntactically correct. If they are not, a message
    should explain in as much details as possible what is wrong. All
    submitted files can be downloaded.
  \item Give the possibility to use a \emph{token\/} against a particular
    submission (see next).
  \item View a detailed feedback of the evaluation of a submission in
    two cases: for tasks that are configured to have detailed
    feedback, and for submissions against which the user has used a
    token.
  \item Interact with the contest managers, in particular for asking
    questions concerning tasks' statements.
  \end{squishlist}

  \subsection{Evaluation Server}

  Evaluation Server's goal is to transform submissions in evaluated
  submissions, and to transform the set of evaluated submissions in a
  ranking (the current ranking of the contest).

  More precisely, these are its duties.
  \begin{squishlist}
  \item \ES{} maintains a queue of jobs to be assigned to some
    \WS{}. A job is of two types. The first is the
    \emph{compilation\/} of a submission (and the evaluation on a
    small set of testcases to let the user know that the submission is
    syntactically correct). The second is the complete
    \emph{evaluation\/} of a submission on the whole testcases set.
  \item \ES{} decides the priority assigned to jobs: compilation heve
    top priority, followed by the evaluation of a submission for which
    the contestant has used a token, followed by all other
    evaluations.
  \item A job is put in the queue When \CWS{} notifies that a
    submission has been placed; at start, \ES{} look at already places
    submissions to ensure that all of them are evaluated; also, the
    contest managers can ask the \ES{} to reevaluate some submissions.
  \item When a job is queued and when \WS{} finishes some job, \ES{}
    update one or more \emph{views\/} of the contest. A view is a
    snapshot of some part of the state of the contest integrated with
    some derived data, saved in \CDBS{}; it is used by other services
    so they do not need to compute the view several time.
  \item When a \WS{} finishes an evaluation, it compute the
    \emph{outcome\/} of the submission and, if necessary, updates the
    outcomes of all the other submissions (this is needed for example
    when the outcome depends on the best submission so far).
  \end{squishlist}

  \subsection{Worker Server}

  Worker Server's aim is to do the low-level job of \ES{}. That is, it
  has to take care of compiling the submissions and executing them
  against the testcases, assigning to each of them their evaluation.

  \subsection{Ranking Web Server}

  Ranking Web Server is the interface of the contest to the
  world. Typically, it shows the live ranking of the contest, maybe
  skipping some problems, or hiding the data after a certain amount of
  time.

  \subsection{Logging Web Server}

  Logging Web Server is the interface of the system to the contest
  managers. It shows the complete state of the contest, including the
  state of all services (load, memory usage, length of the queues,
  etc.). It gives some control over the other services; for example,
  contest managers can ask \ES{} to evaluate again some submissions.

  \section{Data structures}

  Here we describe the four main classes that describe the structure
  and the inner working of a contest.

  \subsection{The \id{Contest} class}

  \begin{squishlist}

  \item \id{name} (string): the name used to refer to the contest;

  \item \id{description} (string): a textual description of the
    task;

  \item \id{tasks} (list): the tasks composing this contest; each
    element of the list is the CouchDB id of a \id{Task} object;

  \item \id{users} (list): the contestants taking part to the
    contest; each element of the list is the CouchDB id of a
    \id{User} object;

  \item \id{token\_total\_num} (integer): the total number of tokens
    that a contestant can use during the whole contest; a value of
    $-1$ means that the limit is not enforced;

  \item \id{token\_num} (integer): the initial number of tokens
    available to a contestant for all tasks;

  \item \id{token\_max\_num} (integer): the maximum number of tokens
    available to a contestant for all tasks at a given time; a value
    of $-1$ means that the limit is not enforced;

  \item \id{token\_min\_interval} (integer): the minimum interval (in
    minutes) between using two tokens;

  \item \id{token\_gen\_time} (integer): the time (in minutes) after
    which a new token is made available to the contestant; more
    precisely, each \id{token\_gen\_time} minutes a new token is
    available if there are less than \id{token\_max\_num} tokens
    available;

  \item \id{start} (integer): the UNIX timestamp of the beginning
    of the contest; it's not set when setting up the contest, but when
    scheduling its timings;

  \item \id{stop} (integer): the UNIX timestamp of the end of the
    contest; like \id{start}, it's set when scheduling the contest's
    timings;

  \item \id{submissions} (list): the submissions issued during the
    contest; it's empty at the beginning of the contest and gets
    filled while the contest is running; each element of the list if
    the CouchDB id of a \id{Submission} object.

  \end{squishlist}

  \subsection{The \id{User} class}

  \begin{squishlist}

  \item \id{username} (string): the identifier used to authenticate
    the contestant to the contest system;

  \item \id{password} (string): the password used to authenticate the
    contestant to the contest system;

  \item \id{real\_name} (string): the real name of the contestant;

  \item \id{ip} (string): the IP address (in dotted quad notation) of
    the machine used by the contestant; it's used to check that
    requests for this contestant come from the correct machine and
    avoid cheating; this check is not enforced when \id{ip} is set to
    the null address (``\texttt{0.0.0.0}''); currently there is no
    support for checking against IPv6 addresses;

  \item \id{tokens} (list): the tokens used by this contestant; each
    element of the list if the CouchDB id of a \id{Submission} object,
    on which the contestant used a token;

  \item \id{hidden} (boolean): when this flag is set the user is not
    shown on public rankings; this flag in meant to be used for
    testing accounts (for example, the managers of the contest could
    add a test user to check if the system is correctly receiving and
    validating submissions).

  \end{squishlist}

  \subsection{The \id{Task} class}

  \begin{squishlist}

  \item \id{name} (string): the name used to refer to the task;

  \item \id{title} (string): a title for this task;

  \item \id{attachments} (dictionary): the files attached to the
    statement of the problem (example code snippets, files to
    complement the problem description, \dots); in each entry the key
    is the filename and each value is the \FS{} id of an attachment;

  \item \id{statement} (string): the \FS{} id of the statement PDF,
    that describes the task to the contestant;

  \item \id{time\_limit} (float): the timeout for this task (in
    seconds);

  \item \id{memory\_limit} (float): the memory limit for this task (in
    megabytes);

  \item \id{task\_type} (integer): the type of this task; how
    submissions for this task gets compiled, executed and checked
    depend on this value via a plugin system;

  \item \id{submission\_format} (list): the description of the files
    expected within a submission for this task; each element of the
    list is the name of a file that must exist in the submission, with
    the language extension replaced with ``\texttt{\%l}'';

  \item \id{managers} (dictionary): the files provided by the contest
    organizers used when compiling and evaluating contestants'
    submissions; its use depends on the \id{task\_type}; in each entry
    the key is the filename and each value is the \FS{} id of a
    manager;

  \item \id{score\_type} (integer): the type of scoring method for
    this task; how the score for this task is assigned to the
    contestants depends on this value via a plugin system;

  \item \id{score\_parameters} (dictionary): the parameters passed to
    the scoring system for this task; its meaning depends on the value
    of \id{score\_type};

  \item \id{testcases} (list): the description of the testcases for
    this task; each element of the list describes a case, but its
    meaning depends on the value of \id{task\_type};

  \item \id{public\_testcases} (list): the testcases that contestants
    are able to see even without using a token; each element is an
    index in the \id{testcases} list;

  \item \id{token\_total\_num} (integer), \id{token\_max\_num}
    (integer), \id{token\_num} (integer), \id{token\_min\_interval}
    (integer), \id{token\_gen\_time} (integer): these parameters have
    the same meaning of those present in \id{Contest}, but are
    specific for this task.

  \end{squishlist}

  \subsection{The \id{Submission} class}

  \begin{squishlist}

  \item \id{user} (string): the contestant that made this submission;
    it is the CouchDB id of a \id{User} object;

  \item \id{task} (string): the task for which this submission was
    filed; it is the CouchDB id of a \id{Task} object;

  \item \id{timestamp} (integer): the UNIX timestamp of the instant in
    which this submission was issued;

  \item \id{files} (dictionary): the files submitted by the
    contestant; in each entry the key is the filename and each value
    is the \FS{} id of a file;

  \item \id{executables} (dictionary): the compiled files for this
    submission; in each entry the key is the filename and each value
    is the \FS{} id of an executable, but its meaning depends on the
    task (some tasks could even have no executables at all, like
    output-only tasks); this value is set when compiling the
    submission;

  \item \id{compilation\_outcome} (string): the description of the
    compilation outcome, used by the \ES{} and \CWS{} to know whether
    the submission has to be evaluated or not; it's content should be
    the string ``ok'' if the compilation was successful (and, thus,
    the submission accepted), or ``fail'' if there were errors that
    made the submission to be rejected; this value is set when
    compiling the submission;

  \item \id{evaluation\_outcome} (list): the description of the
    outcome obtained by this submission on each testcase; each element
    of the list describes the outcome on the corresponding testcase,
    but its format and meaning depends on the specific task; this
    value is used by the scoring function to compute the score for all
    the contestants; this value is set when evaluating the submission;

  \item \id{compilation\_text} (string): a textual and human-readable
    description of the outcome of the compilation, shown to the
    contestant; this value is set when compiling the submission;

  \item \id{evaluation\_text} (list): a textual and human-readable
    description of the outcome of the evaluation, shown to the
    contestant if a token was used on this submission; each element of
    the list refers to the corresponding testcase; this value is set
    when evaluating the submission;

  \item \id{token\_timestamp} (integer): the UNIX timestamp of the
    instant when the token was used for this submission; not set if no
    token was used;

  \end{squishlist}

  \section{Internals}

  Here we describe the RPC method offered by the services, and some
  implementation details.

  \section{Terms}

  Here we describe all terms that have a specific meaning inside \CMS{}.

  \begin{squishlist}
  \item[Importer (program).] An importer is a program that fills the
    configuration of the contest and \FS{} with the data needed for
    the contest. Contest managers may decide to do this step by hand.
  \item[Compilation (process).] It is the process done by \WS{}, where
    it compiles the submission and runs it against a small number of
    testcases to ensure that it is syntactically correct. The result
    of the compilation is always visible to the contestant, and if the
    submission didn't pass the compilation, a message should be
    visible to hint at the part to correct. The procedure used to
    compile and run the submission is specified by the task type, and
    usually makes use of one or more managers.
  \item[Evaluation (process).] It is the process done by \WS{}, where
    it compiles the submission and runs it against a large number of
    testcases; for each testcase, \WS{} produces an evaluation.
  \item[Evalutation (data).] \WS{} produces a numerical value for each
    submission and each testcase, called evaluation; this is not the
    correct value to use to build the ranking of the contest. The
    evaluation is given by a manager of the task.
  \item[Outcome (data).] The outcome is a numerical value assigned by
    \ES{} to a submission, that depends on the evaluations of all
    testcases and all submissions sent before. The method used to
    compute the outcome is determined by the outcome type. This is the
    value to use to build the rank of the contest.
  \item[Manager (source code)] Managers are source code provided by
    the contest managers, that is going to be linked to the source
    code provided by the contestants, or used to evaluate the output
    of the submission after its execution against a testcase.
  \item[Task type (class).] A task type is a class that specifies how
    to treat the files submitted by the contestant and the managers
    provided by the contest managers; usually, this turns out to be
    instructions for the compilation and linking of source code into
    one or more executable, and the instructions for the execution of
    these programs. A second duty of the task type is to decide how to
    merge a submission with the last one (sometimes it is useful to
    allow partial submissions, e.g. in output-only tasks).
  \item[Outcome type (class).] An outcome type is a class that
    specifyies how to translate the evaluations of the submissions
    into outcomes. Its behaviour is influenced by parameters
    sepecified in the configuration of the task.
  \item[View (data).] A collection of derived data that is useful to
    more than one service and maybe heavy to compute. It is saved in
    \CDBS{} and available for all other services. For example, the
    outcome, and the rankings.
  \item[Token.] A contestant can use a token against one of his
    submissions, in order to have an overview of the goodness of the
    submission. The actual result that the contestant can see is
    specified by the configuration. At the beginning of the contest,
    each contestant has a certain number of tokens; between two usage
    of a token, some time has to pass; moreover token regenerates
    after some time. These configuration may be tasks-oriented or
    contest-oriented
  \end{squishlist}
\end{multicols}
\end{document}
