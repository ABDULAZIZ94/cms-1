\documentclass[a4paper,8pt]{amsart}

\usepackage[osf,sc]{mathpazo}
\usepackage[a4paper]{geometry}
\usepackage{fullpage}
\usepackage{multicol}

\newcommand{\CMS}{\textsc{cms}}

\newcommand{\FS}{\textsc{fs}}
\newcommand{\CDBS}{\textsc{cdbs}}
\newcommand{\CWS}{\textsc{cws}}
\newcommand{\ES}{\textsc{es}}
\newcommand{\WS}{\textsc{ws}}
\newcommand{\LWS}{\textsc{lws}}
\newcommand{\RWS}{\textsc{rws}}

\newenvironment{squishlist}{%
  \begin{list}{\textbullet}%
    { \setlength{\itemsep}{0pt}%
      \setlength{\parsep}{3pt}%
      \setlength{\topsep}{3pt}%
      \setlength{\partopsep}{0pt}%
      \setlength{\leftmargin}{1.5em}%
      \setlength{\labelwidth}{1em}%
      \setlength{\labelsep}{0.5em} }%
}{\end{list}}

\title{A Contest Management System for programming contests}
\author{Matteo Boscariol, Massimo Cairo, Stefano Maggiolo, Giovanni
  Mascellani}
\date{\today}


\begin{document}

\maketitle
\tableofcontents

\begin{multicols}{2}

  \section{Introduction}

  When organizing a programming contest, there are three main stages:
  \begin{squishlist}
  \item the first is to develop all the data that the assigned tasks
    need (i.e., the statement, solutions, testcases, information on
    how to grade submissions, etc.);
  \item the second, that happens when the contest is onsite, is to
    properly configure the computers that the contestants are going to
    use during the contest, in particular with respect to network
    security;
  \item the third is to manage the actual contest (accepting and
    grading submissions, give feedback on them, display a live
    ranking, etc.).
  \end{squishlist}

  The aim of this project is to give a good answer to the third
  problem. We want to develop a contest management system that is
  secure, extendable, adaptable to different situations, and easy to
  use.

  \section{General structure of the system}

  The system is organized in a modular way, with different services
  running (potentially) on different computers, and providing
  extendability via service replications on several computers.

  The system is composed of the following functional pieces:
  \begin{squishlist}
  \item File Storage (\FS{}), that stores files that can be accessed from
    the rest of the system;
  \item CouchDB Server (\CDBS{}), that stores the configuration and the
    state of the contest;
  \item Contest Web Server (\CWS{}), that interacts with the contestants;
  \item Evaluation Server (\ES{}), that grades the submissions and keeps
    track of the state of the contest;
  \item Worker Server (\WS{}), that actually runs the submissions;
  \item Ranking Web Server (\RWS{}), that expose in real time to the
    world what we want the world to know about the contest;
  \item Log Web Server (\LWS{}), that expose to the contest managers
    everything about the contest, and gives the possibility to change
    on the run some configurations.
  \end{squishlist}

  One is not forced to run everything; here are some examples.
  \begin{squishlist}
  \item Bare: there are only a \FS{}, a \CDBS{}, and a \CWS{}; the
    system would accept submissions, but is unable to give feedback to
    the users and to the world, lacking something that evaluate the
    submissions.
  \item Minimal: add an \ES{} and a \WS{}; now we can evaluate
    submissions, so the users are informed of the state of the
    submissions.
  \item Complete: add a \RWS{} and a \LWS{}, in order to show the live
    ranking and controls locally if anything strange is happening.
  \item Redundant: add several other \WS{}'s to manage big contest
    without having long evaluation queues.
  \end{squishlist}

  \subsection{Interaction with the task development}

  The development of the tasks is completely detached from the
  system. One can use some directory structure, ease the development
  with the usage of makefiles, common libraries, or other
  techniques. We believe that every contest manager has his own
  preferred way on how to proceed in this stage, but we want him to
  use our \CMS{} anyway.

  Therefore, there is only one point of contact between our \CMS{} and
  the actual structure on the contest manager's file system: anytime
  after the finalization of the tasks and before the start of the
  contest, the contest manager has to run an \emph{importer\/} that
  configures \CMS{} and fills \FS{} with the needed files.

  We wrote an example of importer that works for the structure used by
  the Italian IOI team. Converting this program to use other
  structures should be easy.

  A slightly more complicated issue is that different contest managers
  can use different type of tasks.

  \subsubsection{An example}

  Before IOI 2010 in Canada, the tasks assigned at the IOI were mainly
  "batch" tasks, where the contestants have to submit a stand-alone
  source file; there had been sporadic cases of "output-only" tasks,
  where the contestants have the input cases and need to submit only
  the outputs to these inputs, and of "interactive" tasks, where the
  contestants submit a source file that will be linked to a given
  library.

  From IOI 2010, the distinction between batch and interactive tasks
  vanished, being them united in the "programming" tasks. In all these
  tasks, the contestants have to submit one or more source files that
  implement one or more functions; these source files will be linked
  to a given source file that implements the I/O and calls the
  contestants' functions.

  Note that the programming tasks may need to be threated differently
  in the evaluation. For example, when the statement of the task says
  that the contestants have to implement two functions that cannot
  communicate between each other, the evaluation needs to ensure this.

  These difference can be implemented *inside* the system, as
  plugins. We believe that we internal structure of a task is flexible
  enough to manage most task types.

  \section{Description of the services}

  We give here a detailed description of the capabilities and the
  current role of the services that compose \CMS{}.

  \subsection{Generalities}

  We distinguish two kind of services in \CMS{}: external services and
  local services. An external service is something that serves web
  pages; there are three of them: \CWS{}, that serves the contestants,
  \LWS{}, that serves the contest managers, and \RWS{}, that serves
  everyone else. A local service instead is not accessible from
  outside the system. Except \CDBS{}, which actually did not need any
  line of code from us, all else are local services.

  The services are written in Python; external services use Tornado to
  serve dynamic web pages, and nginx to serve static files and do the
  caching. All services communicate between each other using the
  \texttt{SimpleXMLRPCServer} library.

  \subsection{File Storage}

  File Storage is a simple service, but is essential to any
  contest. It is capable of accepting files, storing them and offering
  them when asked to.

  In particular, these are its uses:
  \begin{squishlist}
  \item before the contest, the importer ask \FS{} to store the tasks'
    data: the statement in PDF (needed by \CWS{}), inputs, outputs,
    source files that are going to be used during the evaluation of
    submissions (needed by \WS{});
  \item during the contest, after receiving a submission, \CWS{} ask
    \FS{} to store its associated files (being them outputs or source
    files); then, \WS{} needs these files to evaluate the submission.
  \end{squishlist}

  \subsection{CouchDB Server}

  CouchDB Server is not actually a service of \CMS{}, but no more that
  an istance of CouchDB. Therefore, it can stores \emph{documents\/},
  that for us are serializations of Python objects, where a reference
  to another object is translated to the CouchDB id of the referenced
  object.

  Its main use in \CMS{} is to store the contest's data, that is the
  specification of the contest, tasks, contestants, submissions. The
  documents in \CDBS{} may have links to files contained in \FS{}.

  \subsection{Contest Web Server}

  Contest Web Server is a Tornado application that serve (localized)
  web pages to the contestants. Here are the main functionality
  exposed to them.
  \begin{squishlist}
  \item View the tasks' statement and the attachments (that usually
    are example of implementations of the functions, or big example of
    test data).
  \item Submit a (partial) solution for a task. Since a task may
    require more than one file, if \CWS{} receives an archive (zip,
    tar, tar.gzip, or tar.bzip2), it unpack the archive and assign the
    files to the submission. It is a duty of the plugin specifying the
    task type to decide what happens when there are some files
    missing. For a programming task, it makes sense to reject the
    submission; for an output-only task, it makes sense to use the
    previously sent outputs. When a submission is received, \CWS{}
    informs \ES{} and ask it to evaluate the submission.
  \item View the state of submissions, in particular if they have been
    compiled and executed against the inputs that ensure that the
    submission is syntactically correct. If they are not, a message
    should explain in as much details as possible what is wrong. All
    submitted files can be downloaded.
  \item Give the possibility to use a \emph{token\/} against a particular
    submission (see next).
  \item View a detailed feedback of the evaluation of a submission in
    two cases: for tasks that are configured to have detailed
    feedback, and for submissions against which the user has used a
    token.
  \item Interact with the contest managers, in particular for asking
    questions concerning tasks' statements.
  \end{squishlist}

  \subsection{Evaluation Server}

  Evaluation Server's goal is to transform submissions in evaluated
  submissions, and to transform the set of evaluated submissions in a
  ranking (the current ranking of the contest).

  More precisely, these are its duties.
  \begin{squishlist}
  \item \ES{} maintains a queue of jobs to be assigned to some
    \WS{}. A job is of two types. The first is the
    \emph{compilation\/} of a submission (and the evaluation on a
    small set of testcases to let the user know that the submission is
    syntactically correct). The second is the complete
    \emph{evaluation\/} of a submission on the whole testcases set.
  \item \ES{} decides the priority assigned to jobs: compilation heve
    top priority, followed by the evaluation of a submission for which
    the contestant has used a token, followed by all other
    evaluations.
  \item A job is put in the queue When \CWS{} notifies that a
    submission has been placed; at start, \ES{} look at already places
    submissions to ensure that all of them are evaluated; also, the
    contest managers can ask the \ES{} to reevaluate some submissions.
  \item When a job is queued and when \WS{} finishes some job, \ES{}
    update one or more \emph{views\/} of the contest. A view is a
    snapshot of some part of the state of the contest integrated with
    some derived data, saved in \CDBS{}; it is used by other services
    so they do not need to compute the view several time.
  \item When a \WS{} finishes an evaluation, it compute the
    \emph{outcome\/} of the submission and, if necessary, updates the
    outcomes of all the other submissions (this is needed for example
    when the outcome depends on the best submission so far).
  \end{squishlist}

  \subsection{Worker Server}

  Worker Server's aim is to do the low-level job of \ES{}. That is, it
  has to take care of compiling the submissions and executing them
  against the testcases, assigning to each of them their evaluation.

  \subsection{Ranking Web Server}

  Ranking Web Server is the interface of the contest to the
  world. Typically, it shows the live ranking of the contest, maybe
  skipping some problems, or hiding the data after a certain amount of
  time.

  \subsection{Logging Web Server}

  Logging Web Server is the interface of the system to the contest
  managers. It shows the complete state of the contest, including the
  state of all services (load, memory usage, length of the queues,
  etc.). It gives some control over the other services; for example,
  contest managers can ask \ES{} to evaluate again some submissions.

  \section{Internals}

  Here we describe the RPC method offered by the services, and some
  implementation details.

  \section{Terms}

  Here we describe all terms that have a specific meaning inside \CMS{}.

\end{multicols}
\end{document}
